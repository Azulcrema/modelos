\documentclass{article}
%\usepackage[spanish]{babel}
\usepackage[utf8]{inputenc}
\usepackage{graphicx}

\title{Ecuaciones en diferencias}
\author{Nayeli Ruiz Chablé}
\date{18 de Septiembre de 2017}
\begin{document}
\maketitle
\section{Ecuaciones diferenciales de primer grado.}

\subsection{Ecuaciones lineales.}

Una ecuación lineal en diferencias de primer grado tiene la forma: $x_{n+1}=ax_n$, donde $a$ es una constante.
 
La fórmula para resolver ecuaciones lineales es:
\begin{equation}
  \label{eq:1}  
 x_n=a^nx_0
\end{equation}

Por ejemplo,si iniciamos una inversion con 1000 pesos con un interés mensual del 1\% ,obtenemos lo siguiente:

\includetgraphics[width=8cm]inversion.png}

\subsection{Ecuaciones afines.}

Una ecuación afin en diferencias de primer grado tiene la forma $x_{n+1}=ax_n+b$, donde $a$ y $b$ son constantes.
 
\begin{equation}
  \label{eq:2}  
  x_n=a^n(x_0-\alpha)+\alpha
\end{equation}
donde $\alpha=\frac{b}{1-a}$.

Para deducir esta fórmula usamos $$\sum_{i=0}^{n-1}a^i=\frac{a^n-1}{a-1}$$


\section{Ecuaciones de segundo grado.}

El método para resolver estas ecuaciones está inspirado en la fórmula \ref{eq:1}
\end{document}

